% Created 2016-01-25 Mon 15:26
\documentclass[12pt]{article}
\usepackage[top=1in, bottom=1.in, left=1in, right=1in]{geometry}
  \usepackage[font={small}]{caption}
\usepackage{setspace}
\usepackage{bibentry}

\usepackage[utf8]{inputenc}
\usepackage{lmodern}
\usepackage[T1]{fontenc}
\usepackage{fixltx2e}
\usepackage{graphicx}
\usepackage{longtable}
\usepackage{float}
\usepackage{wrapfig}
\usepackage{rotating}
\usepackage[normalem]{ulem}
\usepackage{amsmath}
\usepackage{textcomp}
\usepackage{marvosym}
\usepackage{wasysym}
\usepackage{amssymb}
\usepackage{amsmath}
\usepackage[version=3]{mhchem}
\usepackage[numbers,super,sort&compress]{natbib}
\usepackage{natmove}
\usepackage{url}
\usepackage{minted}
\usepackage{underscore}
\usepackage[linktocpage,pdfstartview=FitH,colorlinks,
linkcolor=blue,anchorcolor=blue,
citecolor=blue,filecolor=blue,menucolor=blue,urlcolor=blue]{hyperref}
\usepackage{attachfile}
\author{John Kitchin}
\date{\today}
\title{REU Supplement for grant CBET1506770}
\begin{document}

\setstretch{1.21}
\section{Student involvement in the research project}
\label{sec-1}

Our grant title "Modeling Bulk Composition Dependent Alloy Surface Properties Under Reaction Conditions" addresses several challenges on that topic. One of the challenges is how to effectively model segregation in alloy systems. We are requesting an REU supplement to support an undergraduate student to work on modeling alloy segregation using a new modeling approach we are developing in this work. The undergraduate will work with one PhD student, and one MS student to use a new atomistic potential we developed to model segregation in Cu-Pd alloys.



\section{PI experience with undergraduate research}
\label{sec-2}
The PI has mentored 21 undergraduates in research projects while at Carnegie Mellon over the last decade. Thirteen of the students were women and one of them was an underrepresented minority supported by a PREM-REU program at CMU. Four other students were also supported by an REU program at CMU. Three of the undergraduate students have gone to graduate school to earn PhDs in chemical engineering. the rest have obtained employment in a STEM field.

\section{Process and criteria for selecting the student}
\label{sec-3}

A student has not yet been identified for the project. On approval of the supplement the PI will reach out to colleagues (Maria Curet-Arana, Maria M Martinez Inesta and Yomaira J. Pagan Torres) at the University of Puerto Rico Mayagüez to recruit an underrepresented, domestic minority student for the project. If a student cannot be identified through this avenue, a local student at Carnegie Mellon University will be supported. In either case, a resume and statement of interest will be requested from applicants. This will be used to identify qualified students who are likely to benefit from the experience. Preference will be given to applicants who express interest in going to graduate school in a STEM field.

\newpage
\bibliographystyle{unsrtnat}

nobibliography:<replace: bibfiles>


\section{Broader impacts of the work}
\label{sec-4}
% Emacs 25.1.50.1 (Org mode 8.2.10)
\end{document}